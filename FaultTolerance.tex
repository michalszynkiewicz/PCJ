\documentclass[12pt]{article}
\usepackage[utf8]{inputenc}


\linespread{1.05}

\begin{document}


\title{Fault Tolerance for PCJ}

\author{Michał Szynkiewicz}
\date{2015}
\maketitle
% mstodo: make it more a fairy-tale ;)

\section{Motivation}

Distributed computations are run on lots of nodes and often take large amount of time.
Having a 24 hour long computation on 1000 nodes means almost 3 years computations.

Hence, it is practical to make a parallel computing library fault tolerant.

\section{Dictionary}
\begin{itemize}
\item \textbf{PCJ library} - a library for Java language that helps to perform parallel and distributed calculations, able to work over ethernet or infiniband providing users with the uniform view across nodes.

\item \textbf{node} - a physical machine that takes a part in computations.

\item \textbf{node 0} - distinguished node. This paper assumes it doesn't fail.

\item \textbf{thread} - the smalles unit of computation. Realized by single java thread.
\end{itemize}

\section{Concepts}
Currently we assume that there is a special node, namely node 0, that never dies.
We plan to provide two fault tolerance strategies: ignoring node failure and respawning failed threads
  \\
Moreover we plan to provide a resilient storage to make it possible to restore thread state from a snapshot.
Resilient storage will have two modes: synchronous and asynchronous.
Synchronous mode will allow programs to recover from the latest position achieved by the failed thread.
In asynchronous mode state will be saved to storage in a non-blocking fashion.
Programs using this strategy will work much faster, but the state saved in the storage may be not the latest one.
\\                                  \\
Example use cases of strategies: \\
% mstodo:
\begin{tabular}{|l | p{3.5cm} | p{3.5cm} | p{3.5cm}|}
  \hline
  & No resilient storage & Asynchronous resilient storage & Synchronous resilient storage\\
  \hline
  Ignore &  the failed node is ignored, all its' data is lost. Might be used for instance
                    for monte carlo algorithms, where part of the results can be ignored.
                    &  & \\
  \hline
  Respawn &  a new node is created as a replacement for the failed node. &  a new node is created as a replacement for the failed node.
                Resilient storage is used to snapshot the data and restore computations from latest snapshot.
                     & a new node is created as a replacement for the failed node.
                       Resilient storage is used to snapshot the data and restore computations from latest snapshot.
                       Useful for algorithms which require \\
  \hline
\end{tabular}



\section{Implementation outline}

The immortal node 0 is monitoring other nodes by sending PING message and waiting for PONG.
If, for some node, sending PING fails, or the node does not send PONG for a long time,
it is assumed failed.

When node0 discovers some node is down:
\begin{itemize}
\item for Ignore policy - it populates the information about failure to all nodes.
Each node removes the failed node from its configuration and that node is no longer
taken into account
\item for Respawn policy - for each of threads on the failed node,
it creates a thread on a working physical node. Then the newly created threads
take over the job of failed nodes threads. As in the above case - reconfiguration message
is sent to all nodes, which in turn replace old threads with new ones in their configuration.
\end{itemize}


\section{Current status}
Currently no-resilient-storage, Ignore policy is being implemented.

\section{Performance overhead}

\section{Summary}

\end{document}
